\documentclass[a4paper]{article}
\usepackage{fullpage}
\usepackage{setspace}
\usepackage{graphicx}
\usepackage[dvips]{color}

%\usepackage[bookmarks,dvips,pdfhighlight=/O,pdfstartview=FitH]{hyperref}
%\usepackage[dvips]{graphicx}
\usepackage{amsmath,amssymb,amsopn}
\usepackage{latexsym} \usepackage{bm} \usepackage{mathrsfs}
\usepackage{mathbbol} % \usepackage[dvips]{color}
%\usepackage{pslatex} \usepackage{subfig}
\usepackage{mathtools}
\date{\today}
\title{Notes on Diss. and Heating in Supersonic HD and MHD turbulence}
\author{Andrew Emerick}


\begin{document}
\maketitle

\section{Introduction}
\begin{itemize}
\item Molecular clouds are observed to contain supersonic turbulence, the details
of which can have an effect on the SF ($\vec{B}$ and $\mathcal{M}$)
\item Turbulent driving mechanism is unknown: periodic, pulse, driving scales?
\item Study details of energy dissipation in HD and MHD supersonic turbulence
\item Study effects of numerical dissipation by comparing to ZEUS
\item Compare solenoidal and compressional driving
\item Convergence test for HD and MHD cases
\end{itemize}

\section{Numerical Methods}
\begin{itemize}
\item 64$^{3}$ to 1024$^{3}$ resolution simulations made with ATHENA
\item Ideal, isothermal MHD equations with periodic BC's
\item Exact / approximate Riemann solver for HD / MHD
\end{itemize}
\subsection{Set up}
\begin{itemize}
\item Uniform stationary medium with $\bar{\rho}$=1
\item Magnetic field is uniform in x-axis, set by the $\beta$ parameter
\item For decaying turbulence, start with single pulse (adiabatic)
\item $\beta$ ranges from 0.02 to infinity in the strong field and HD case
\item Assuming L = 2 pc, n$_{H_2}$ = 10$^{3}$, and T = 10 K (a typical MC),
$\dot{E}$ = 0.4 L$_{\odot}$ and B = 44 $\mu G$ when $\beta$ = 0.02
\end{itemize}

\section{Convergence Results}
For HD:
\begin{itemize}
\item HD convergence by 64$^{3}$ or 128$^{3}$ depending on driving scale and
energy injection
\item solenoidal / compressive fraction converges to 0.78 / 0.22 in HD
\item energy dissipation timescale is independent of driving scale and $<$ 1
\item $\mathcal{M}$ converges to 7.2 to 5.8 for large / small driving scales
\end{itemize}

For MHD:
\begin{itemize}
\item Higher mach number for larger driving scales, though smaller than HD in
all cases
\item Convergence is around 512$^{3}$, with sol / comp frac at 0.9 / 0.1
\item dissiption timescale is around unity
\end{itemize}

There is a good agreement between overall diagnostics for ATHENA and ZEUS. 
Rapid decay of supersonic turbulence is not due to high numerical dissipation

\section{$\mathcal{M}$}
Mach number has a strong effect on the turbulence state. Though E$_C$ is small
in all cases, it increases slightly from 0.05 to 0.1 with higher mach number.
Dissipation timescale decreases with mach number, from 1.7 at low numbers to
less than one at high mach numbers.

\section{Power Spectra}
Much higher resolution is required to appropriately resolve the intertial 
range of the simulation. For the HD case, supersonic turbulence has a roughly
power law slope in the intertial range (between -2 and -1.7) but that higher
resolution runs are needed to appropriately resolve this. Bottleneck effect
is present in the supersonic turbulence PS.

Power law in the MHD case is roughly -4/3. \textbf{u} should scale to the 
-5/3, but is observed here at -4/3. This is expected if the helicity
timesacle is non-negligible

For the MHD system, anisotropy in the PS is present, with more perpendicular
to the mean field (-4/3 perp, -2 parallel). Compressive comp. is isotropic. 
Results suggest that turbulent cascade in the compressive comp. of 
velocity is not present here.

\section{Sonic Scale}
Sonic scale is point at which velocity dipserions become subsonic. For the 
HD case, it scales as l to the 0.58. There is no nice power law relationship
for the MHD case, as physics change drastically on many scales due to 
different wave families of the MHD equations 
\section{Questions}
\begin{itemize}
\item Why did they start with a uniform B field? What motivates that the 
magnetic field has to be uniform and coherent on scales of the MC? This should
likely have a large effect on the dissipation rate and development of the 
subsequent system.
\end{itemize}

\section{Conclusions}
\begin{itemize}
\item Supersonic turbulence decays rapidly (often far less than flow time)
\item High res is needed to appropriately model the intertial range
\item Less than high res is needed for convergence in overall diagnostics
\item Should not rely on PS to characterize these states until higher res is
readily achievable
\item MHD turbulence is very anisotropic (roughly HD parallel, shallow perp)
\item Need structure functions to determine origin of power laws 
\item Larger driving scales lead to greater time variability in temperature

\end{itemize}


\end{document}
