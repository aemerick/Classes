\documentclass{article}
\usepackage{graphicx}

\begin{document}
\title{Proposed Numerical Project}
\author{Andrew Emerick}

\date\today
\maketitle

\section{Idea 1:}
For my numerical project this semester, I propose to write a simple code to 
evolve the Euler equations in 1D. Specifically, I will use my scheme to evolve
the two Sod test cases as given in Ch. 18 of Computational Gasdynamics.
Currently, the exact scheme is open ended, but I was thinking of coding up
and testing one of the predictor corrector methods (18.3 or 18.4/18.15). 
The latter set of equations sound more interesting, which is MacCormack's
method, alternating between FTBS and FTFS methods at each time step, and 
employing a fixed CFL condition.

\section{Idea 2:}
Similar idea, different method. Looking back at some of my MHD research in
Minnesota, the code I used (of which I frankly understood very little of what
was going on beneath the hood) used something based on a 2nd order 
Roe-type upwind scheme to evolve the MHD equations.
Maybe I could do the Sod tests using section 18.3.2 (Roe's first order 
upwind method). Glancing at the section, this seems substantially more involved,
but also probably more beneficial.


\end{document}
