\documentclass[12pt]{article}% uses letterpaper by default

%---------- Uncomment one of them ------------------------------
\usepackage[includeheadfoot, top=.5in, bottom=.5in, hmargin=1in]{geometry}

% \usepackage[a5paper, landscape, twocolumn, twoside,
%    left=2cm, hmarginratio=2:1, includemp, marginparwidth=43pt,
%    bottom=1cm, foot=.7cm, includefoot, textheight=11cm, heightrounded,
%    columnsep=1cm, dvips,  verbose]{geometry}
%---------------------------------------------------------------
\usepackage{fancyhdr}
\usepackage{verbatim}
\usepackage{url}
\pagestyle{fancy}
\usepackage[latin1]{inputenc}
\usepackage{amsmath}
\usepackage[pdftex]{graphicx}
\usepackage[english]{babel}
\usepackage{amsfonts}
\usepackage{amssymb}
\usepackage{setspace}
\usepackage{}
%\doublespacing
\singlespacing

\newcommand{\degrees}{\ensuremath{^\circ}}
\newcommand{\arcmin}{\ensuremath{'}}
\newcommand{\arcsec}{\ensuremath{"}}
\newcommand{\hours}{\ensuremath{^\mathrm{h}}}
\newcommand{\minutes}{\ensuremath{^\mathrm{m}}}
\newcommand{\seconds}{\ensuremath{^\mathrm{s}}}


\rhead{Emerick 2014}
\renewcommand{\rightmark}{}


\begin{document}



\section{Problem 1}

Fig.~\ref{fig:p1} gives the results to problem 1.

\begin{figure}[tb]
\center
\includegraphics[width=0.45\linewidth]{dp_to}
\includegraphics[width=0.45\linewidth]{dp_te}
\vspace{0.1cm}
\includegraphics[width=0.45\linewidth]{to-te}
\caption{}
\label{fig:p1}
\end{figure}

\section{Problem 4}

The evolution of the ionization fraction as a functin of temperature is given in Fig.~\ref{fig:p4}. Recombination is defined as the point at which $x$ = 0.5, and is given as T$_{rec}$ = 3740 K for $\eta$ = 5.5$\times$10$^{-10}$, as shown by the solid black line in the plot. Photon decoupling and the surface of last scattering occur at the same time/redshift/temperature, as the surface of last scattering exists because photons decoupled from matter. This occurs when H = $\Gamma$, which is at a redshift of z = 1100 and a temerature of about T = 3000 K, shown by the black dashed line. For $\eta$ = 5.5$\times$ 10$^{-10}$, this occurs at x = 4.33 $\times$ 10$^{-3}$.


\begin{figure}[t]
\center
\includegraphics[width=0.9\linewidth]{x_T}
\caption{Ionization fraction of Hydrogen as a function of temperature in the universe.}
\label{fig:p4}
\end{figure}




\end{document}
