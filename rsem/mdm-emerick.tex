% MDMPROP.TEX -- MDM electronic proposal form.
% Revision: September 4, 2003 --- valid for the Spring 2014 observing semester.
% DEADLINE: 5:00 PM Thursday Oct. 31, 2013
%
% General Instructions:
%
%   I. Where/how/when to submit this form:
%
%      Give a paper copy to Jules or his mailbox on the 10th floor
%      of Pupin by the above deadline, or email the postscript or
%      pdf file to jules@astro.columbia.edu.
%
% THE FORM STARTS HERE

% please don't modify or delete this line.
\documentstyle[mdmprop,11pt]{article}



\begin{document}

% Give a descriptive title for the proposal in the \title command.
%
%    \title{TEXT}
%
% Note that a title can be quite long; LaTeX will break the title into
% separate lines automatically.  If you wish to indicate line breaks
% yourself, do so with a `\\' command at the appropriate point in
% the title text.

\title{Uncovering the Origin of Warm Gas in Galaxy Clusters: Identifying Possible
       Galactic Origins of Cluster Lyman Alpha Absorbers}

% Investigator's (PI and CoI) information blocks.
%
% The PI must be affiliated with Columbia University.
% Please give each investigator's name, email address.
% At least one investigator must be listed as definitely
% being present at the telescope.
%
%    \name{OBSERVER NAME}
%    \emailaddress{EMAIL ADDRESS}
%    \atthetelescope{Y/N}
%
% DO NOT remove the \begin{PI} and \end{PI} or the \begin{CoI} and
% \end{CoI} lines.

\begin{PI}
\name{Andrew Emerick}			% REQUIRED
\emailaddress{emerick@astro.columbia.edu}		% REQUIRED
\atthetelescope{Yes}       % REQUIRED
\end{PI}

\begin{CoI}
\name{Mary Putman}			% REQUIRED
\emailaddress{mputman@astro.columbia.edu }		% REQUIRED
\atthetelescope{No}       % REQUIRED  
\end{CoI}

\begin{CoI}
\name{Greg Bryan}			% REQUIRED
\emailaddress{gbryan@astro.columbia.edu}		% REQUIRED
\atthetelescope{No}       % REQUIRED
\end{CoI}

%\begin{CoI}
%\name{}			% REQUIRED
%\emailaddress{}		% REQUIRED
%\atthetelescope{}       % REQUIRED
%\end{CoI}

% You can supply more CoI blocks, but only the first three will be
% printed on the form.

% Give a general abstract of the scientific justification appropriate for
% a non-specialist.  Write the abstract between the \begin{abstract} and
% \end{abstract} lines.  Limit yourself to approximately 175 words.

% DO NOT remove the \begin{abstract} and \end{abstract} lines.

\begin{abstract}
Galaxy clusters are the largest gravitationally bound structures in the Universe,
and are host to a variety of gas flow phenomena. We want to be able to map out 
the warm gas component (T$\sim$10$^{4-5}$~K) of a galaxy cluster's intracluster
medium (ICM), which traces processes such as gas accretion onto the cluster
from filaments, and interactions between the ICM and the cluster's galaxies. 
This has been done recently through Ly$\alpha$ absorption observations in the
Virgo cluster with HST's COS. However, being able to discriminate against
or determine the possible galactic origins of the warm gas component requires
a complete understanding of the velocities of galaxies within a few hundred
kpc (in projection) of a given line of sight. In light of upcoming observations
of Ly$\alpha$ absorption in the Coma Cluster, we propose to obtain more 
precise velocity measurements of galaxies around 9 lines of sight in order to 
better understand the relationship between the observed warm gas and the 
galaxies within the Coma Cluster.
\end{abstract}

% Indicate whether this proposal is part of your PhD thesis work by
% putting a Y or an N inside the \thesis{} curly braces.

\thesis{N}

% Indicate whether you are requesting long-term status by putting
% a Y or an N inside the \longterm{} curly braces.  If you do want
% long-term status (a "Y" answer), please tell us the number of nights,
% number of semesters, and what telescopes by filling in \longtermdetails{}.
% Please be brief; \longtermdetails is limited to one line.

\longterm{N}
\longtermdetails{}

% List the details of the observing runs being requested, for
% UP TO THREE runs.  The parameters for each run are segregated
% between \begin{obsrun} and \end{obsrun} lines.  Please be sure
% that the information is isolated properly for each run.
%
%   \telescope{}	For example, \telescope{1.3m}
%   \instrument{}	For example, \instrument{Mk III + Charlotte}
%   \lunardays{}	For example, \lunardays{14}
%   \optimaldates{}	For example, \optimaldates{Sep 1 - Nov 30}
%   \acceptabledates{}	For example, \acceptabledates{Sep 1 - Dec 15}
%
%
% Instrument combinations may be specified with "+".
%
% \numnights should give the number of nights of the run
% \lunardays should specify the maximum number of nights from new moon 
% which can be utilized to accomplish your scientific goals.  It should 
% be a number from 1 (new moon) to 14 (full moon).
%
% \optimaldates should contain the range of OPTIMAL dates.
%
% \acceptabledates should give the range of ACCEPTABLE dates (i.e., you
% would not accept time outside those limits).
%
% To enter the acceptable and optimal date ranges, please use two 
% dash-separated dates with 3-letter abbreviations for the month
% (Jan, Feb, Mar, Apr, May, Jun, Jul, Aug, Sep, Oct, Nov, Dec) 
% followed by the day.  For example:  \optimaldates{Feb 15 - Apr 23}
%
% If you need to enter two or more ranges of acceptable or optimal dates
% for a single observing run, separate the ranges by commas.  For example:
% \acceptabledates{Oct 7 - Nov 24, Jan 7 - Jan 31}
% It is not necessary to give date ranges based on lunar phase information.
% For instance, if you wish to observe an object in March through May
% within five days of new moon, you may give \lunardays{5} and 
% \optimaldates{Mar 1 - May 31} instead of multiple shorter date ranges.
% 
% DO NOT remove the \begin{obsrun} and \end{obsrun} blocks.

\begin{obsrun}
\telescope{2.4}
\instrument{OSMOS, VPH grism}
\numnights{3}
\lunardays{}
\optimaldates{March 18-22 ; April 17-20}
\acceptabledates{March 13-22; April 6-20}
\end{obsrun}

\begin{obsrun}
\telescope{}
\instrument{}
\numnights{}
\lunardays{}
\optimaldates{}
\acceptabledates{}
\end{obsrun}

\begin{obsrun}
\telescope{}
\instrument{}
\numnights{}
\lunardays{}
\optimaldates{}
\acceptabledates{}
\end{obsrun}

% You may NOT supply more obsrun blocks.  Three is the limit.

% If there are dates that you cannot use for non-astronomical reasons,
% (i.e., other than moon phase or when your object is up)
% please give the dates by filling in the curly braces in \unusabledates{}.
% Please be brief; \unusabledates is LIMITED TO ONE LINE.

\unusabledates{}

% In the following "essay question" sections, the delimiting pieces of
% markup (\justification, \feasibility, etc.) act as LaTeX \section*{}
% commands.  If the author wanted to have numbered subsections within
% any of these, LaTeX's \subsection could be used.

% SCIENTIFIC JUSTIFICATION
%
% Give the scientific justification for the proposed observations.
% This section should consist of paragraphs of text (and may include
% EPS figures) that follow the \justification line.
% Try to include an explanation of the overall significance to astronomy.

% In order to include an EPS plot, you should use the LaTeX "figure"
% environment.  The plot file is included with the \plotone{FILENAME}
% command; two side-by-side plot files can be included by typing
% \plottwo{FILENAME1}{FILENAME2}.  Use \caption{} to specify a caption.
% The \epsscale{} command can be used to scale \plotone plots if they
% appear too large on the printed page.
%
% \begin{figure}
% \epsscale{0.85}
% \plotone{sample.eps}
% \caption{Sample figure showing important results.}
% \end{figure}
%
% If you need to rotate or make other transformations to a figure, you may
% use the \plotfiddle command:
% \plotfiddle{PSFILE}{VSIZE}{ROTANG}{HSCALE}{VSCALE}{HTRANS}{VTRANS}
% \plotfiddle{sample.eps}{2.6in}{-90.}{32.}{32.}{-250}{225}
% where HSCALE and VSCALE are percentages and HTRANS and VTRANS are 
% in PostScript units, 72 PS units = 1 inch.
%
% If you wish to use the "reference" environment, follow
% the following example:
%
%\begin{references}
%\reference Armandroff \& Massey 1991 AJ 102, 927.
%\reference Berkhuijsen \& Humphreys 1989 A\&A 214, 68.
%\reference Massey 1993 in Massive Stars: Their Lives in the Interstellar
%  Medium (Review), ed. J. P. Cassinelli and E. B. Churchwell, p. 168.
%\reference Massey, Armandroff, \& Pyke 1993, in prep.
%\end{references}

\justification
Ly$\alpha$ absorption seen through the Ly$\alpha$ forest probes the diffuse gas
contained within the intergalactic medium (IGM) (Rauch 1998). UV spectrographs
aboard the Hubble Space Telescope, most recently the 
Cosmic Origins Spectrograph (COS), provide a window into Ly$\alpha$ absorption 
in the nearby Universe. In addition to its usefulness probing baryons in the
IGM, Ly$\alpha$ can be used to observe warm gas flows into and out of 
galaxy clusters. Clusters are hosts to massive dark matter potential wells
(10$^{14-15}$ M$_{\odot}$) that can shock-heat infalling cool gas to the virial
temperature of the intracluster medium (T$\sim$10$^{7-8}$K). In general, warm
gas (T$\sim$10$^{4-5}$~K) traces the inflow of gas from pristine filamentary
structures, as well as processes unique to cluster environments, shock as the
ram-pressure stripping of gas in infalling galaxies, galactic starvation, 
and galaxy-galaxy interactions. Recently, Yoon et. al. 2012 (hereafter Y12) used
COS to systematically probe the warm gas distribution in
galaxy clusters through 14 lines of sight to background QSOs through the Virgo
cluster. Y12 indicated the presence of an inflowing warm gas component that could
have galactic or filamentary origins. Regardless, this compenent traces important dynamical
processes associated with galaxy clusters. We propose to conduct observations
to discriminate against or determine the possible galactic orgins of this Ly$\alpha$
absorbing gas seen in recent and upcoming lines of sight towards
richer and more massive Coma Cluster.

The investigators have secured the necessary time with COS to observe nine 
lines of sight towards background QSOs through the Coma Cluster. Fig. 1
gives these lines of sight, spread throughout the cluster environment, along
with the SDSS galaxies shown in gray.
From the results of Y12, we expect a covering
fraction of unity for Ly$\alpha$ absorbing gas at N$_{HI}$ above 10$^{13.1}$ cm$^{-2}$.
Since we expect a more massive galaxy cluster such as Coma to contain more 
inflowing warm gas, each of our lines of sight should exhibit some Ly$\alpha$ 
absorption associated with the Coma Cluster. The circumgalactic medium (CGM)
of galaxies in the field show up in Ly$\alpha$ absorption
as long as their impact parameter to the line of sight is within a few hundred kpc
(Prochaska et. al. 2011). However, cluster environments affect the evolution of 
galaxies and tend to suppress the CGM; this suppresses the observable galactic
Ly$\alpha$ absorption (Yoon \& Putman 2013). For this reason, we restrict our
target list to galaxies within a few hundred kpc of each line of sight.
Yoon \& Putman 2013 studied the 
association of galaxies and Ly$\alpha$ absorbers within the Virgo Cluster, 
developing a method to pair Ly$\alpha$ absorbers with galaxies. However, this
analysis requires a complete understanding of the distribution and velocities
of galaxies around each line of sight. Although the Coma Cluster is well studied,
and many galaxies already have measured velocities, there are still many near
each line of sight with no velocity measurements. We propose to use MDM to obtain
velocity measurements of a selection of galaxies near each line of sight in
order to extend the analysis of Yoon \& Putman 2013 to the Coma Cluster. This 
is important in order to better understand how a diverse range of cluster 
environments affect their resident galaxies.

% In addition,
%an important discriminator of the possible galactic origin of the absorbers
%is metallicity, indicating whether or not the gas is enriched and likely
%galactic in origin, or pristine, likely from the IGM filaments. Using
%the spectral observations, we can obtain an understanding of the metallicity and
%star formation properties of the galaxies near each line of sight, which
%can be a useful indicator also of whether or not the galaxy has undergone 
%a stripping or starvation event. \textbf{need to describe the specific obsevations
%needed for this (i.e. which line ratios, etc.)}

Currently, we are conducting synthetic Ly$\alpha$ observations through a galaxy
cluster simulated with the adaptive mesh refinement code Enzo
(The Enzo Collaboration et. al. 2013).
In this work, we reconstruct the three dimensional distrubution of the
observable warm gas within a galaxy cluster in order to determine the origin
of the observed absorption features. The synthesis of this 
computational work, the observations of Ly$\alpha$ absorption in Virgo and Coma,
and the observations proposed here will provide a robust test of our theoretical
understanding of gas flows in galaxy clusters.
The MDM observations are essential, then, towards determining the potentially 
galactic origin of the observed warm gas, and testing what we predict from
cosmological simulations. These observations would provide a better understanding
of the relationship between galaxies and their cluster environments, and the 
evolution of the gaseous baryons within galaxy clusters. 

\begin{figure}
\label{fig:coma}
\epsscale{0.85}
\plotone{figures/coma_QSO_los.eps}
\caption{The Coma Cluster with the distrubution of associated galaxies from SDSS (gray dots).
The red diamonds give the lines of sight to the QSO's being observed with HST. The dashed
circle gives the Coma Cluster virial radius, 2.9 Mpc. The lines of sight as listed
in Table 1 are ordered in increasing distance from cluster center.}
\end{figure}



%The origin of these Ly$\alpha$ absorbers is not well understood. The absorbers themselves
%could originate from ram-pressure stripping events as galaxies fall into a cluster's 
%potential well, or could trace cooler gas acretting onto the galaxy cluster from 
%IGM filaments. In addition, the gas could come from galactic outflows from galaxies associated
%with the galaxy cluster. To discriminate between these possibilities, the metallicities of these
%absorbers must be better constrained, and the properties of galaxies along the line of sight
%of the absorbers must be well understood.

%In light of upcoming observations using HST's COS of lines of sight towards 9 background quasars
%through the Coma Cluster, we propose to obtain spectra of \textbf{number} galaxies
%near these lines of sight. 

%Currently, computational work is being performed by the authors using the adaptive mesh refinement
%code Enzo (Bryan et. al 2013) to conduct synthetic UV spectra observations on a Virgo-like and a
%Coma-like cluster. These results will be combined with the previous Virgo observations 
%in Y12, and the upcoming Coma observations in order to provide a solid theoretical understanding
%of the origin of these absorbers. Using this in combination with the proposed MDM observations
%has the potential to test our predictions for the origins of this warm gas component in
%galaxy clusters in general, and also to potentially pinpoint the specific galaxies from which 
%these absorbers came from. Doing so would allow for a better understanding of the relationship
%between galaxies and their cluster environment, or the accretion process by which matter
%falls into the cluster potential from cosmic filaments.



\begin{references}
\vspace{-0.2cm}
\reference The Enzo Collaboration et. al. 2014, ApJS, 211:19
\reference Prochaska, J. X., Weiner, B., Chen, H.-W., Mulchaey, J., \& Cooksey, K., 2011, ApJ, 740, 91
\reference Rauch, M. 1998, ARA\&A, 36, 267
\reference Yoon, J. H., Putman, M.E., Thom, C., Checn, H.-W., \& Bryan, G. L. 2012 ApJ, 754, 84
\reference Yoon, J. H., Putman, M.E., 2013, ApJ, 722, L29
\reference Yoon, J. H., Schawinski, K., Sheen, Y-K., Ree, C. H., \& Yi, S. K., 2008, ApJS, 176, 414 
\end{references}

% FEASIBILITY
%
% Assess the technical and scientific feasibility of the observations.
% This section should consist of text and tables only (no figures)
% following the \feasibility line.
%
% List objects, coordinates, and magnitudes (or surface brightness,
% if appropriate), desired S/N, wavelength coverage and resolution.
% Justify the number of nights requested as well as the specific
% telescope(s), instruments, and lunar phase.  Indicate the optimal
% detector, as well as acceptable alternates.  If you've requested
% long-term status, explain why this is necessary for successful
% completion of the science.

\feasibility
We propose to measure the velocity of galaxies given
in Table 1. For each galaxy, Table 1 shows its position (RA,DEC), the nearby
background QSO, the galaxy's magnitude, and the separation (in arcmin)
from the line of sight. At Coma's
redshift, z = 0.023, 10$'$ corresponds roughly to 250 kpc. Ultimately, there are
many galaxies within 10$'$ of each line of sight without known velocity measurements.
However, we restrict ourselves to the brightest galaxies that are closest
to each of the lines of sight. This selection is fairly arbitrary, but restricts our sample
to galaxies most likely associated with the Coma Cluster, and galaxies that would 
most likely be responsible for an observed absorption feature. Some galaxies near
the line of sight are likely associated with the Coma Cluster, but only have 
photometric redshifts available. We restrict ourselves to five galaxies near each
sight line shown in Fig. 1, placing precedence
on those in Coma with photometric redshifts available.

The velocity of these galaxies can be readily determined using the
OSMOS detector on the 2.4m telescope. This  instrument was designed specifically
for the purpose of studying galaxies in galaxy clusters in the nearby Universe.
OSMOS is perfect for observing galaxies within Coma. Given the magntitudes listed
in Table 1, we can make our observations within 3 hours of exposure time. 
Given the possible range
of slit dimensions, 0.6-20$"$ long and 0.6-1.2$"$ wide, we will be able to 
fit all of our galaxies on a single mask (45 slits in all) without collisions
between slits.

We will use the VPH grism to conduct observations in two sets, over the
wavelength range 310-590nm (peak at 500 nm) and 310-680nm (peak at 640nm). Given
the required exposure time per observation, this implies six hours of exposure
to make the required measurements. The VPH grism has a peak resolution of
R=1600, which would allow velocity measurements 
accurate to about 180 km/s at H$\alpha$. The accuracy of velocity measurements are improved
greatly by measuring the Doppler shift for two or three lines. With OSMOS,
we will be able to observe H$\alpha$ (656.28 nm), H$\beta$ (486.13 nm), H$\delta$
(410.17 nm), and the Ca H and K lines (396.85 nm and 393.37 nm). However,
H$\alpha$ and H$\beta$ will likely be the most useful (nearest two to the 
resolution peaks at 500 nm and 640 nm). For older, elliptical galaxies are not 
likely to be seen in emission, but rather absorption. For this case, we can
use NaD (589.2 nm), Mgb (517.5 nm) and also potentially the Ca H and K lines to
measure the velocity.

%\begin{itemize}
%\item Some notes:
%\item slit length: 0.6-20 arcsec
%\item slit width: 0.6-1.2 arcsec
%\item Good performance over 380-1000 nm with triple prism
%\item R=400 -$>$ 60 over 400nm to 1000 nm
%\item 40 min exposure was enough for (i-z)=3.29,i=23.30 QSO at z=6.43
%\end{itemize}


\begin{table}
\small
\begin{tabular}{ c c c c c c c }
\hline
Object Name & RA & DEC & Sep. (") & Mag.\\
\hline
& \textbf{HB89.1259+281}&&&\\
\hline
SDSS J130129.58+275052.1 & 195.37329 & 27.84782 & 0.400 & 18.2r \\
SDSS J130128.71+275134.8 & 195.36963 & 27.85968 & 0.488 & 19.9r \\
SDSS J130132.58+275018.0 & 195.38578 & 27.83835 & 1.273 & 18.6g\\
ABELL 1656* & 195.2308 & 27.79202 & 8.081 & 20.4g\\
SDSS J130123.95+275316.6* & 195.34983 & 27.88796 & 2.357 & 21.0g\\
\hline
& \textbf{HB89.1258+285} &&&\\
\hline
SDSS J130101.38+282011.4 & 195.25579 & 28.33652 & 0.459 & 19.5r\\
APD2000 J130104.3+281956 & 195.26808 & 28.33228 & 0.783 & 18.9R\\
SDSS J130055.49+281939.7 & 195.23121 & 28.32772 & 1.191 & 19.4r\\
GALEXASC J130051.96+281852.4 & 195.21663 & 28.31486 & 2.136 & 16.9R\\
SDSS J130056.49+281650.0 & 195.23539 & 28.28056 & 3.068 & 18.5r\\
\hline
& \textbf{TON0694} &&&\\
\hline
%SDSS J130313.13+281114.1 & 13 03 13.1 & +28 11 14 & TON0694 & 0.108 & 20.6g\\ % 21.06i
%SDSS J130311.00+281124.9 & 13 03 11.0 & +28 11 25 & TON0694 & 0.529 & 21.7g\\ % 20.69i
%SDSS J130315.43+281041.4* & 13 03 15.4 & +28 10 41 & TON0694 & 0.689 & 20.8g \\ % 20.09i
%SDSS J130312.63+281151.4 & 13 03 14.7 & +28 11 49 & TON0694 & 0.732 & 22.9g \\ %  20.473
%SDSS J130308.38+281047.7* & 13 03 35.9 & +28 10 48 & TON0694 & 1.075' & 21.8g\\ % 21.68
SDSS J130313.13+281114.1 & 195.80472 & 28.18726 & 0.108 & 20.6g\\
SDSS J130308.38+281047.7* & 195.78494 & 28.17993 & 1.075 & 21.8g\\
SDSS J130311.68+280832.5 & 195.79868 & 28.14238 & 2.604 & 19.65\\
SDSS J130300.41+280945.8 & 195.75171 & 28.16275 & 3.096 & 19.81\\
SDSS J130302.68+280852.0 & 195.76118 & 28.1478 & 3.21 & 19.8g\\
\hline
& \textbf{RX.J1303.7+2633} &&&\\
\hline
SDSS J130346.40+263327.3 & 195.94334 & 26.55759 & 0.239 & 21.8g\\
SDSS J130339.80+263329.8 & 195.91585 & 26.5583 & 1.406 & 18.9g\\
SDSS J130351.64+263228.9 & 195.96518 & 26.54138  & 1.473 & 20.8g\\
SDSS J130340.05+263028.6 & 195.91692 & 26.50796  & 3.06 & 19.81\\
SDSS J130359.02+263425.4 & 195.99593 & 26.57372  & 3.149 & 19.8g\\
\hline
& \textbf{FBQS.J1252+2913} &&&\\
\hline
SDSS J125327.13+291330.0 & 193.36306 & 29.22505  & 0.49 & 21.7g\\
SDSS J125320.37+291237.7 & 193.3346 & 29.21048  & 1.24 & 20.5g\\
SDSS J125334.07+291232.6 & 193.39197 & 29.20907 & 2.141 & 19.2g\\
SDSS J125314.44+291556.2 & 193.31019 & 29.26562 & 3.459 & 19.5g\\
SDSS J125314.37+291602.6 & 193.3099 & 29.26741 & 3.55 & 19.2g\\
\hline
& \textbf{TON0133} &&&\\
\hline
SDSS J125100.27+302637.8 & 192.75115 & 30.44385 & 0.933 & 19.1g\\
SDSS J125054.85+302541.9 & 192.72856 & 30.42832 & 1.176 & 19.9g\\
SDSS J125105.99+302715.4 & 192.77499 & 30.4543 & 1.984 & 19.6g\\
SDSS J125106.96+302743.2 & 192.779 & 30.46201 & 2.48 & 19.2g\\
SDSS J125020.07+302310.9 & 192.58363 & 30.38638 & 9.033 & 18.7g\\
\hline
& \textbf{FBQS.J130451.4+245445} &&&\\
\hline
SDSS J130459.32+245548.0 & 196.24717 & 24.93003 & 2.071 & 19.3g\\
SDSS J130445.27+245719.2 & 196.18866 & 24.95536 & 2.91 & 19.9g\\
NGP9 F379-0188786 & 196.194 & 24.84016 & 4.492 & 19.16g\\
SDSS J130507.51+245517.3 196.2813 & 24.92148 & 3.688 & 19.8g\\
SDSS J130512.51+245242.6 & 196.30213 & 24.87852 & 5.207 & 18.8g\\
\hline
& \textbf{SDSS.J130429.04+311308.2} &&&\\
\hline
SDSS J130428.67+311311.2 & 196.11947 & 31.21979 & 0.092 & 22.0g\\
MAPS-NGP O\_323\_0147666 & 196.12412 & 31.27387 & 3.299 & 18.35\\
SDSS J130431.67+311725.5 & 196.13199 & 31.29042 & 4.325 & 19.72\\
SDSS J130412.89+311357.4 & 196.05373 & 31.23264 & 3.546 & 19.9g\\
GALEXASC J130415.22+311713.9 & 196.06322 & 31.28699 & 5.043 & 18.5\\
\hline
& \textbf{SDSSJ125846.67+242739.1} &&&\\
\hline
1RXS J125847.1+242740 & 194.69625 & 24.46139 & 0.105 & 17.8\\
SDSS J125846.50+242746.5 & 194.69376 & 24.46293 & 0.128 & 21.5g\\
SDSS J125845.71+242840.8 & 194.69048 & 24.47802 & 1.05 & 19.8g\\
SDSS J125840.10+242821.3 & 194.66709 & 24.47259 & 1.649 & 19.9g\\
SDSS J125847.38+243153.8 & 194.69745 & 24.53164 & 4.248 & 19.2g\\
\hline
\end{tabular}
\caption{Given are the positions of proposed galaxies to observe, along with the 
background QSO each galaxy is near, the separation in arcmin from the line of sight
to the QSO, and galaxy magnitude in the designated band. Galaxies marked with *
have a photometric redshifts available.}
\end{table}


% OTHER FACILITIES
%
% Why MDM? If you are using other facilities for this project,
% explain how the MDM observations fit into the scheme of things.
%
% This section should consist of text and tables only (no figures)
% following the \feasibility line.

\whymdm
HST's COS is the only other observational instrument being used for this project.
The time on HST has already been obtained, and all observations will be made by
the end of summer 2014. The MDM observations are essential for identifying the
possible galactic origin of the identified Ly$\alpha$ absorbers in the Coma 
Cluster. Since the investigators are affiliated with Columbia University, the ease
of access to MDM facilities makes it an attractive option for conducting this
experiment. More importantly, the availability of the mutli-object spectrograph
OSMOS, designed to look at the objects we propose to observe, makes MDM the 
logical choice.

% PAST USE
%
% List your allocation of telescope time at MDM during the
% past 3 years, and describe the status of the project (cite
% publications where appropriate).  Mark any allocations of time
% related to the current proposal with a \relatedwork{} command.
% Are your MDM observations achieving their goals?

\thepast
AE has not had any previous use of MDM.

\end{document}

