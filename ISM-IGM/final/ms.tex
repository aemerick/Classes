\documentclass[a4paper]{article}
\usepackage{fullpage}
\usepackage{setspace}
\usepackage{graphicx}
\usepackage[dvips]{color}

%\usepackage[bookmarks,dvips,pdfhighlight=/O,pdfstartview=FitH]{hyperref}
%\usepackage[dvips]{graphicx}
\usepackage{amsmath,amssymb,amsopn}
\usepackage{latexsym} \usepackage{bm} \usepackage{mathrsfs}
\usepackage{mathbbol} % \usepackage[dvips]{color}
%\usepackage{pslatex} \usepackage{subfig}
\usepackage{mathtools}
\date{\today}
\title{ISM/IGM Final}
\author{Andrew Emerick}


\begin{document}
\maketitle
\section{Molecules in Clouds and Photodissociation Regions}

\subsection{A}
Carbon is present as C$^{+}$ in neutral Hydrogen (HI) regions because it has a lower
ionization threshold (11.26 eV) than HI (13.6 eV). The dominant form of ionization
is UV photoionization, so this is the case as long as A$_{V} <$ 3.
This HI region should be optically
thick to UV radiation between 13.6 and 54.4 eV, meaning both H and He will be
neutral. Carbon should be the most abundant ionized atomic species. In this case,
it is reasonable to assume that the main contributor towards the electron
population is from the ionization of carbon, thus n$_{e} \sim$ n$_{\text{C}^+}$ 
is a reasonable approximation.


\subsection{B}
The lowest two vibrational/rotational energy levels for H$_{2}$ have wavelengths of 
28.18 $\mu$m and 17.03 $\mu$m. These transitions correspond to a temperatures
of 510.8 K and 845.3 K respectively. Assuming these states could be readily 
excited, these transitions allow the cooling of a molecular cloud down to
510.8 K. Below this temperature, CO becomes essential. Its J(1-0) transition
has $\lambda$ = 2.6 mm, and thus a corresponding temperature of 5.54 K. Although
H$_{2}$ is essential for cooling in general, CO cooling becomes vital in 
the formation of molecular clouds, which have temperatures around 10 - 20 K.

\subsection{C}
As given in Draine pg. 368, the linewidth $\sigma_v$ in a virialized cloud
is given as $<\sigma_v^2>$ = 6GM/5L, where L is the characteristic size
of the cloud (L = 2R). Assuming that R$\propto \sigma_v^2$, then the
gas surface density, $\Sigma$ is given as
\begin{equation}
\Sigma = \frac{M}{A} \propto \frac{M}{R^2} \propto \frac{\frac{5\sigma_v^2L}{6G}}{0.5L\sigma_v^{2}} \propto \frac{\sigma_v^2L}{\sigma_v^2L} = \text{constant}.
\end{equation}
Thus, the gas surface density is constant assuming a virialized molecular cloud.

For L$_{\text{CO}}$ = $\left(\sqrt{2\pi}T_{B}\sigma_v\right)\pi R^2$ and 
M$_{\text{gas}}$ = $\Sigma \pi R^2$, the relationship between CO luminosity and 
gas mass can be readily seen to be
\begin{equation}
\frac{M_{\text{gas}}}{L_{\text{CO}}} = \frac{\Sigma}{\sqrt{2\pi}T_{B}\sigma_{v}}
\end{equation}

This relationship is then a function of easily measureable quantities,
the brightness temperature of the CO line (T$_{B}$) and the linewidth $\sigma_v$,
with the surface gas density being a constant. There is then a linear 
relationship between gas mass and CO luminosity.

\subsection{D}
The CR ionization rate as traced by N(H$_3^+$)/N(H$_2$) is given as
\begin{equation}
\xi \approx (k_{16.9} + k_{16.10})n_{H}x_e\frac{N(H_3^+)}{2N(H_2)}(1+\phi_s)^{-1}
\end{equation}
where $\phi_s$ is given as
\begin{equation}
\phi_s = \left(1-\frac{x_e}{1.2}\right)\left(\frac{0.67}{1+\left(x_e/0.05\right)}\right)
\end{equation}

x$_{e}$ can be estimated from the ionized fraction of Carbon as x$_{e}$=x(C$^+$).
For the provided numbers, the range of possible ionization rates is
1.28$\times$10$^{-18}$s$^{-1} < \xi <$ 1.28$\times$10$^{-16}$s$^{-1}$. The values
used in Draine vary, but are in the neighborhood of 1.0 to 6.0 $\times$10$^{-16}$s$^{-1}$.
These sit at or above the given upper limit. 

\subsection{E}
The isothermal sound speed for cool (T$\sim$10 K) gas expected of molecular
clouds is on the order of 1 km/s. For shocks colliding at roughly twice the 
velocity dispersion of the MW's disk, the shock types are expected to be
either C or C*, but not J, in a MHD two-fluid shock (as relevant for molecular
clouds). This implies that there is no viscous jump between the upstream and 
downstream gas. In the C* case, the neutrals smoothly transition from supersonic
to subsonic pre-shock, then back to supersonic as the gas cools in the post
shock region. These cases are valid assuming a subalfvenic ion velocity. Two-fluid
MHD shocks are not expected to dissociate H$_2$ for shock speeds below 40 km/s. 
In this case then, H$_2$ cooling is significant, as its vibrational/rotational
states will be excited in the shock heating of the molecular cloud.

In the shocked regions, the resulting heating may allow for reaction 33.24 to take
place, converting C$^+$ and H$_2$ to CH$^+$ and H with the input of 0.40 eV. This
requires gas heating above 10$^3$ K. Draine \& Katz (1986) model the chemical
evolution MHD shocks as applied to molecular clouds, and found that C-type 
shocks in these clouds could generate CH$^+$. They compare their results to 
three molecular clouds ($\xi$ Per, $o$ Per, and $\chi$ Oph) and are able
to construct models that produce the observed CH$^+$ (with other
observational constraints) reasonably well. These models required shock 
velocities between 7 and 12 km/s.

\subsection{F}
Assuming a uniform density molecular cloud, an ionization and shock front that
propogates through the cloud resulting from the formation of a star will switch
between a few different forms during its journey outward. The initial ionization
front will be a strong, R-type front which propogates outward at speeds much
greater than twice the isothermal sound speed (much greater than roughly 20 km/s).
Once the front reaches speeds close to twice the sound speed, usually when 
R$\sim$R$_{\text{SO}}$, the front generates a shock wave ahead of itself. 
The ionization front continues to move through this shocked (heated and 
accelerated gas), but does so now as a weak D-type shock. The ionization front
is now trapped behind the hydrodynamical shock front, as it becomes subsonic.
They both continue to travel outward, but decelerate over time.

The D-type shock propogating outwards causes ionized gas to flow outward
at roughly the sound speed. There is a large pressure jump between the ionized
and neutral gas (P is larger in the nuetral gas). This pressure jump is responsible
for accelerating ionized gas away from the ionization front (i.e. it pushes
the ionized gas outward).


\section{Supernovae and Winds in the ISM}

\subsection{A}
The mass flux of the expanding wind is readily given as
$\dot{M}$ = 4$\pi$r$^{2}\rho$v$_w$. In order for the velocity of the wind to
be constant as the wind expands, the density profile must go as r$^{-2}$.

\subsection{B}
The typical velocity dispersion of an elliptical galaxy is on the order of
200 km s$^{-1}$. Assuming an AGB wind velocity appreciable to this velocity
(just taking v$_w$ = 200 km s$^{-1}$), Eq. 36.28 in Draine gives the temperature
of a postshock region for strong shocks. This equation requires knowledge of the
mean molecular weight of the gas, which can range from 0.609 m$_H$ to 1.273 m$_H$
for ionized and neutral gas respectively. Taking $\mu$ as a free parameter,
the temperature of the post shock gas ranges from 5.5$\times$10$^{5}$ K to 
1.16$\times$10$^{6}$ K for fully ionized and neutral gas respectively. This
temperature range corresponds to wavelengths of 26 nm to 12.4 nm, both of 
which are in the FUV/UV range. Cooling radiation from this gas should
be at wavelengths at and below FUV.

\subsection{C}
Using a massive O star for the computation, an O star has wind velocities
between 1500 and 2500 km/s, and mass loss rates between 10$^{-6.5}$ and 
10$^{-5}$ M$_{\odot}$ yr$^{-1}$. For the sake of this computation, I will
take V$_{w}$ = 2000 km/s and M$_{\odot}$ = 5$\times$10$^{-6}$ M$_{\odot}$ yr$^{-1}$.
The rate of mechanical energy deposit in the wind is given as
$\dot{E}$ = $\dot{M}$V$_{w}^2$/2 . Assuming this wind deposits mechanical 
energy at this rate for the entirety of the O star lifetime, this should
give an upper limit for the deposited mechanical energy. It is certainly
the case that the onset of radiative cooling will remove some of the 
deposited energy. The lifetime of an O star is around 10 Myr, but they quickly
progress through the main sequence during that time. I will take 5 Myr as
the lifetime of this hot wind, which gives a total deposited mechanical energy
of 0.995 E$_{51}$ erg, which I will just round to 1.0 E$_{51}$ erg. From 
Draine, the typical energy deposited by a 
Type II supernova will have E$_{51} \approx$ 1 to 20
erg. Given my rough estimate, the energy deposited by the wind is comprable
to but certainly less than the energy deposited by the supernova. 

\subsection{D}
Frankly, I am not terribly sure of how to properly estimate this. The 
accelerated ISM will be swept up during the Sedov-Taylor phase of the evolution,
and also during the subequent snowplow phase (mainly in the latter as the ST
phase ends after the swept up mass is comprable to the mass of the stellar
ejecta). 

Anyway, taking the simple approach and setting 
E$_{\text{ej}}$ = M$_{\text{ej}}$V$^2_{\text{esc}}$/2, where M$_{\text{ej}}$ is 
the total swept up mass that can escape the galaxy, and E$_{\text{ej}}$ is the
energy of that mass. Assuming that 100\% of the supernova energy goes into
pushing the ISM outward (in reality, the effeciency is less than 100\% due
in part to radiative losses), the total amount of mass than can be ejected
by a supernova with E$_{51}$ = 1 and V$_{esc}$ = 500 km/s is 402 M$_{\odot}$. 

\subsection{E}
Using Table 42.1 from Draine, and the text in section 42.3, the supernova rate
for the Milky Way can be readily derived assuming the Chabrier IMF. This is 
given in the text directly as 0.014 yr$^{-1}$. The rate per solar
mass of star formation is given as
\begin{equation}
\frac{\text{Type II SN rate}}{M_{\odot}} = <M>^{-1}\frac{\text{mass}}{\text{total mass}} = 0.011
\end{equation}
where the mass/total mass is taken for stars with M $>$ 8 M$_{\odot}$. Assuming 5\% of these supernova have their gas ejected from the disk, then the amount
of gas ejected per solar mass of star formation is given as 
0.05$\times$0.011$\times$M$_{esc}$, were M$_{esc}$ is the answer from part
D. Thus, using the possibly incorrect answer from Part D,  0.22 M$_{\odot}$ of 
gas is ejected for every 1 M$_{\odot}$ of star formation.

\subsection{F}
The McKee \& Ostriker model gives that the ISM pressure relates to the
supernova rate per unit volume as:
\begin{equation}
\frac{P}{k} = S^{0.77}_{-13}n_{\text{o}}^{-0.15}\times5700~\text{cm}^{-3}~\text{K}.
\end{equation}
Taking S$_{-13}$ = 12.0 (ten times the observed rate in the MW), and n$_{o}$ to
be unity (the mean number density of the ISM), the resulting pressure for a 
star forming galaxy would be P/k = 3.86$\times$10$^{4}$ cm$^{-3}$ K. The 
Bonnor-Ebert mass is given as
\begin{equation}
M_{BE} = 0.26\left(\frac{T}{10 ~K}\right)^{2}\left(\frac{10^6 ~\text{cm}^{-3}~ \text{K}}{P/k}\right)^{1/2} M_{\odot}
\end{equation}
Using the calculated pressure, and assuming a molecular cloud with T$\sim$ 10 - 20 K,
M$_{BE}$ = 1.32 - 5.28 M$_{\odot}$. This is quite large and would seem to imply
that heavy star formation (and thus a large supernova rate) in young galaxies
suppresses supsequent star formation of low mass stars, and favors that of
higher mass. 

\section{Two Phase Interstellar Medium}
See attached

\begin{thebibliography}{9}

\bibitem{orlanski}
  Draine, B.T., Katz, N.
  \textit{Magnetohydrodynamic shocks in diffuse clouds. II - 
          Production of CH(+), OH, CH, and other species}
  ApJ,
  Vol. 310, p. 392-407,
  Nov. 1986

\end{thebibliography}

\end{document}
